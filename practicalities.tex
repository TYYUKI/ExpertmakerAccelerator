
\section{Accelerator Configuration File}

The Accelerator is shipped with a configuration file template with
comments to each line.  It is a good idea to copy and modify this when
a new configuration file is needed.  Below is an example configuration
file that defines two workdirs, called \texttt{import} and
\texttt{processing}.  Both are available for reading, but only the
latter may be written to.  Methods available for use are the
\texttt{standard\_methods} bundled with the Accelerator, and methods
defined in the directory \texttt{dev} (if they are defined in
\texttt{dev/methods.conf}).
\begin{leftbar}
\begin{shell}
workdir=import:${HOME}/accelerator/workdirs/import:16
workdir=processing:${HOME}/accelerator/workdirs/processing:16

target_workdir=processing

source_workdirs=import,processing

method_directories=dev,standard_methods

result_directory=${HOME}/accelerator/results

source_directory=/some/other/path

logfilename=${HOME}/accelerator/daemon.log

py2=/usr/bin/python2.7
py3=/usr/bin/python3.5
\end{shell}
\end{leftbar}
\noindent and here are explanations to all keywords
\starttable

\RP \texttt{workdir} & & A \textsl{workdir}, defined as
\texttt{name:path:slices}.  At least one workdir needs to be defined.
All workdirs used together must have the same number of slices.\\

\RP \texttt{target\_workdir} & & \textsl{workdir} used to
write jobs to.  There can only be one target workdir.\\

\RP \texttt{source\_workdirs} & & a comma separated list of workdirs
available for reading.  These will be the only workdirs that the
Accelerator can ``see''.  \texttt{standard\_methods} is bundled with
the Accelerator and is commonly used.\\

\RP \texttt{method\_directories} && a comma separated list of
directories containing methods.  These will be the only directories
where the Accelerator can ``see'' methods.\\

\RP \texttt{result\_directory} & & A path that is available to all
jobs.  \comment{result\_directory borde finnas med i params.}  \\

\RP \texttt{source\_directory} & & Default root path for
\texttt{csvimport}.  This is to avoid rebuilds of imports if source
files are moved to another directory.  (This typically happens when
seting up a similar system on another physical machine.)\\

\RP \texttt{logfilename} && location of the Accelerator's log.\\

\RP \texttt{py2} and \texttt{py3} && path to Python executables.
\comment{mandatory, are there reasonable defaults?}\\

\stoptable
\comment{ta bort mandatory-kolumnen}

It is possible to assign values in the configuration file using shell
environment variables.  In the example above, workdirs are specified
relative to \texttt{\$\{HOME\}}, for example.  In general, the
assignment is \texttt{\$\{VAR=DEFAULT\}}.

\section{Runner Invocation}
The \texttt{runner} is used to execute build scripts.  it is invocated
like this
\begin{shell}
% automatarunner.py [options] [script]
\end{shell}
assuming the current work directory is the texttt{Accelerator}
directory.  The \texttt{script} is either a filename, or the suffix to
a filename starting with \texttt{automata\_}.
\comment{andra pa detta}

When the \texttt{runner} starts, it will first instruct the
Accelerator to scan all method directories to see if there are any new
or changed methods.  Thereafter, the Accelerator will proceed and scan
all source workdirs to see if any new jobs have been created (by
another Accelerator daemon).  Thereafter, it will execute the build
script.

\begin{tabular}{p{3cm}p{8cm}}
  \texttt{-h}\hspace{1cm}\texttt{--help} & show help message and
  exit.\\[1ex]

  \texttt{-p PORT }\hspace{1cm}\texttt{--port=PORT} & Accelerator
  listening port\\[4ex]

  \texttt{-S SOCKET}\hspace{1cm}\texttt{--socket=SOCKET} & Accelerator
  unix socket (default \texttt{./socket.dir/default})\\[4ex]

  \texttt{-s SCRIPT}\hspace{1cm}\texttt{--script=SCRIPT} & build
  script to run. \texttt{package/automata\_<SCRIPT>.py}.  Defaults to
  ``\texttt{automata}''.  Can be bare arg too.\\[4ex]

  \texttt{-A}\hspace{1cm}\texttt{--abort} & abort (fail) current job(s).\\[1ex]

  \texttt{-q}\hspace{1cm}\texttt{--quick} & skip method updates and
  workdirs checking for new jobs.\\[1ex]
  
\end{tabular}

\begin{verbatim}
Options:
  -h, --help            show this help message and exit
  -p PORT, --port=PORT  framework listening port
  -H HOSTNAME, --hostname=HOSTNAME
                        framework hostname
  -S SOCKET, --socket=SOCKET
                        framework unix socket (default ./socket.dir/default)
  -s SCRIPT, --script=SCRIPT
                        automata script to run. package/[automata_]script.py.
                        default "automata". Can be bare arg too.
  -P PACKAGE, --package=PACKAGE
                        package where to look for script, default all method
                        directories in alphabetical order
  -f FLAGS, --flags=FLAGS
                        comma separated list of flags
  -A, --abort           abort (fail) currently running job(s)
  -q, --quick           skip method updates and checking workdirs for new jobs
  -w, --just_wait       just wait for running job, don't run any automata
  --verbose=VERBOSE     verbosity style {no, status, dots, log}
  --quiet               same as --verbose=no
  --horizon=HORIZON     Time horizon - dates after this are not visible in
                        urd.latest
\end{verbatim}




\section{Daemon Invocation}

\begin{verbatim}
usage: daemon.py [-h] [--debug] [--config CONFIG_FILE]
                 [--port PORT | --socket SOCKET]

optional arguments:
  -h, --help            show this help message and exit
  --debug
  --config CONFIG_FILE  Configuration file (default: ../conf/framework.conf)
  --port PORT           Listen on tcp port (default: None)
  --socket SOCKET       Listen on unix socket (default: socket.dir/default)
\end{verbatim}

\section{Urd Arguments}
