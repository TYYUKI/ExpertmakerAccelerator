
\section{Configuration File}

The Accelerator is shipped with a configuration file template with
comments to each line.  It is a good idea to copy and modify this when
a new configuration file is needed.  Below is an example configuration
file
\begin{leftbar}
\begin{shell}
workdir=import:${HOME}/accelerator/workdirs/import:16
workdir=processing:${HOME}/accelerator/workdirs/processing:16

target_workdir=processing

source_workdirs=import,processing

method_directories=dev,standard_methods

result_directory=${HOME}/accelerator/results

source_directory=/some/other/path

logfilename=${HOME}/accelerator/daemon.log

py2=/usr/bin/python2.7
py3=/usr/bin/python3.5
\end{shell}
\end{leftbar}
\noindent and here are explanations to all keywords
\starttable

\RP \texttt{workdir} & & A \textsl{workdir}, defined as
\texttt{name:path:slices}.  At least one workdir needs to be defined.
All workdirs used together must have the same number of slices.\\

\RP \texttt{target\_workdir} & & \textsl{workdir} used to
write jobs to.  There can only be one target workdir.\\

\RP \texttt{source\_workdirs} & & a comma separated list of workdirs
available for reading.  These will be the only workdirs that the
Accelerator can ``see''.  \texttt{standard\_methods} is bundled with
the Accelerator and is commonly used.\\

\RP \texttt{method\_directories} && a comma separated list of
directories containing methods.  These will be the only directories
where the Accelerator can ``see'' methods.\\

\RP \texttt{result\_directory} & & A path that is available to all
jobs.  \comment{result\_directory borde finnas med i params.}  \\

\RP \texttt{source\_directory} & & Default root path for
\texttt{csvimport}.  This is to avoid rebuilds of imports if source
files are moved to another directory.  (This typically happens when
seting up a similar system on another physical machine.)\\

\RP \texttt{logfilename} && location of the Accelerator's log.\\

\RP \texttt{py2} and \texttt{py3} && path to Python executables.
\comment{mandatory, are there reasonable defaults?}\\

\stoptable
\comment{ta bort mandatory-kolumnen}

It is possible to assign values in the configuration file using shell
environment variables.  In the example above, workdirs are specified
relative to \texttt{\$\{HOME\}}, for example.  In general, the
assignment is \texttt{\$\{VAR=DEFAULT\}}.

\section{Runner Arguments}

\section{Daemon Arguments}

\section{Urd Arguments}
