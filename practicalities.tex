There are three main components in the Accelerator framework, the
\texttt{daemon}, the \texttt{runner}, and \texttt{urd}.  How to
configure and run these components is the topic of this chapter.


\section{Daemon}

\subsection{Invocation}

\begin{verbatim}
daemon.py [-h] [--debug] [--config CONFIG_FILE] [--port PORT | --socket SOCKET]
\end{verbatim}

Optional arguments
\begin{snugshade}
\begin{tabular}{p{4cm}p{9cm}}
  \texttt{-h}\hspace{3cm}\texttt{---help} & show help message and
  exit.\\[4ex]

  \texttt{---debug} & \comment{ga}\\[2ex]
  
  \texttt{---config CONFIG\_FILE} & configuration file, default
  \texttt{../conf/framework.conf}\\[4ex]

  \texttt{---port PORT} & listen on TCP port (default \pyNone)\\[4ex]

  \texttt{---socket SOCKET} & listen on unix socket, default
  \texttt{socket.dir/default}\\[4ex]
\end{tabular}
\end{snugshade}
The Accelerator and Runner will connect using a unix socket by
default.  There is no need to configure anything.  Setting a port will
make communication happen over that port instead.



\subsection{Configuration File}

The Accelerator is shipped with a configuration file template with
comments to each line.  It is a good idea to copy and modify this when
a new configuration file is needed.  Below is an example configuration
file that defines two workdirs, called \texttt{import} and
\texttt{processing}.  Both are available for reading, but only the
latter may be written to.  Methods available for use are the
\texttt{standard\_methods} bundled with the Accelerator, and methods
defined in the directory \texttt{dev} (if they are defined in
\texttt{dev/methods.conf}).
\begin{leftbar}
\begin{shell}
workdir=import:${HOME}/accelerator/workdirs/import:16
workdir=processing:${HOME}/accelerator/workdirs/processing:16

target_workdir=processing

source_workdirs=import,processing

method_directories=dev,standard_methods

result_directory=${HOME}/accelerator/results

source_directory=/some/other/path

logfilename=${HOME}/accelerator/daemon.log

py2=/usr/bin/python2.7
py3=/usr/bin/python3.5
\end{shell}
\end{leftbar}
\noindent and here are explanations to all keywords
\starttabletwo

\RPtwo \texttt{workdir} & A \textsl{workdir}, defined as
\texttt{name:path:slices}.  At least one workdir needs to be defined.
All workdirs used together must have the same number of slices.\\[1ex]

\RPtwo \texttt{target\_workdir} & \textsl{workdir} used to
write jobs to.  There can only be one target workdir.\\[1ex]

\RPtwo \texttt{source\_workdirs} & a comma separated list of workdirs
available for reading.  These will be the only workdirs that the
Accelerator can ``see''.  \texttt{standard\_methods} is bundled with
the Accelerator and is commonly used.\\[1ex]

\RPtwo \texttt{method\_directories} & a comma separated list of
directories containing methods.  These will be the only directories
where the Accelerator can ``see'' methods.\\[1ex]

\RPtwo \texttt{result\_directory} & A path that is available to all
jobs.  \textsl{Currently not in use.}  \\[1ex]

\RPtwo \texttt{source\_directory} & Default root path for
\texttt{csvimport}.  This is to avoid rebuilds of imports if source
files are moved to another directory.  (This typically happens when
seting up a similar system on another physical machine.)\\[1ex]

\RPtwo \texttt{logfilename} & location of the Accelerator's log.\\[1ex]

\RPtwo \texttt{py2} and \texttt{py3} & path to Python executables.
Current versions of the methods in the \texttt{standard\_methods}
directory require \texttt{python2}.
\comment{fill in more details here}\\[1ex]

\stoptabletwo

It is possible to assign values in the configuration file using shell
environment variables.  In the example above, workdirs are specified
relative to \texttt{\$\{HOME\}}, for example.  In general, the
assignment is \texttt{\$\{VAR=DEFAULT\}}.



\clearpage
\section{Runner}

\subsection{Invocation}
The \texttt{runner} is used to execute build scripts.  it is invocated
like this
\begin{shell}
automatarunner.py [options] [script]
\end{shell}
assuming the current work directory is the texttt{Accelerator}
directory.  The \texttt{script} is either a filename, or the suffix to
a filename starting with \texttt{automata\_}.
\comment{andra pa detta}

When the \texttt{runner} starts, it will first instruct the
Accelerator to scan all method directories to see if there are any new
or changed methods.  Thereafter, the Accelerator will proceed and scan
all source workdirs to see if any new jobs have been created (by
another Accelerator daemon).  Thereafter, it will execute the build
script.
\begin{snugshade}
\begin{tabular}{p{4cm}p{9cm}}
  \texttt{-h}\hspace{3cm}\texttt{---help} & show help message and
  exit.\\[4ex]

  \texttt{-p PORT }\hspace{3cm}\texttt{---port=PORT} & Accelerator
  listening port\\[4ex]

  \texttt{-H HOSTNAME}\hspace{3cm}\texttt{---hostname=HOSTNAME} &
  framework hostname\\[4ex]
  
  \texttt{-S SOCKET}\hspace{3cm}\texttt{---socket=SOCKET} &
  Accelerator unix socket (default
  \texttt{./socket.dir/default})\\[4ex]

  \texttt{-s SCRIPT}\hspace{1cm}\texttt{---script=SCRIPT} & build
  script to run. \texttt{package/automata\_<SCRIPT>.py}.  Defaults to
  ``\texttt{automata}''.  Can be bare arg too.\\[4ex]

  \texttt{-A}\hspace{3cm}\texttt{---abort} & abort (fail) current
  job(s).\\[4ex]

  \texttt{-P PACKAGE}\hspace{3cm}\texttt{---package=PACKAGE} & method
  dirs \comment{HUH}\\[4ex]

  \texttt{-f FLAGS}\hspace{3cm}\texttt{---flags=FLAGS} & comma
  separated list of flags\comment{flags}\\[4ex]
  
  \texttt{-q}\hspace{3cm}\texttt{---quick} & skip method updates and
  workdirs checking for new jobs.\\[4ex]

  \texttt{-w}\hspace{3cm}\texttt{---just\_wait} & just wait for running
  job, do not run a build script.\\[4ex]

  \texttt{---verbose=VERBOSE} & verbosity level, one of \texttt{no},
  \texttt{status}. \texttt{dots}, or \texttt{log}.\\[4ex]

  \texttt{---quiet} & same as \texttt{---verbose=no}\\[4ex]

  \texttt{---horizon=HORIZON} & Time horizon - dates after this are
  not visible in \texttt{urd.latest}.\\[4ex]
\end{tabular}
\end{snugshade}


\subsection{Authorization to Urd}
Authorisation to Urd could be set in the \texttt{URD\_AUTH}
environment variable.  A common way to invoke the runner with Urd
authorisation is like this
\begin{shell}
% URD_AUTH=user:passwd ./runner [script]
\end{shell}



\clearpage
\section{Urd}

\subsection{database}

The \texttt{passwd} file.


\subsection{Invocation}
\begin{shell}
urd.py [-h] [--port PORT] [--path PATH]
\end{shell}

\begin{snugshade}
\begin{tabular}{p{4cm}p{9cm}}
  \texttt{-h}\hspace{3cm}\texttt{---help} & show help message and
  exit.\\[4ex]

  \texttt{---port PORT} & listen on TCP port\\[4ex]

  \texttt{---path PATH} & path to database\\[4ex]
\end{tabular}
\end{snugshade}



\section{Workdirs}
\comment{Workdirs are dirs containing a \texttt{slices}-something-something}

Jobdirs will inherit the workdir name and add a suffix that is an
incremental job counter.

\subsection{Creating a Workdir}
To create a new workdir, stop the \texttt{daemon}, add the workdir
information to the configuration file, make it
the \textsl{default\_workdir}, and start the \texttt{daemon} again.


\section{Progress Indication}

During job building, it is possible to press \texttt{C-t},
i.e.\ \texttt{Ctrl} + \texttt{t} simultaneously, to get status
information.  The status will cover the processing state, if it is
in \prepare, \analysis, or \synthesis.  If in \analysis, it will list
all analysis processes that are active at the moment.  If iterating
over a dataset, there will be information about that, including which
dataset in a chain that is currently iterated by which \analysis
process.

Note that \texttt{C-t} could be pressed either in the \texttt{daemon}
or \texttt{runner} shells.
