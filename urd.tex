\section{Introduction to Urd}

Urd is the processing flow controller in the framework.  It is the primary job
dispatcher as well as the bookkeeper of all jobs executed.  Events in
urd are quantified into what is called \textsl{sessions}, and the core
of urd is a transaction log database storing these sessions together
with meta information.  The result is a server providing lookups for
all jobs executed together with their context.

The Urd database is partitioned into what is called \textsl{lists}.
Lists are where information about executed jobs are stored, and
Each user or agent may own one or more lists.
Lists are globally readable, but
writing requires authentication, so that, for example, only
the production user may publish a model to go live.

With the exception of experimental work, all work initiated by urd is
run in closed sessions, with well defined starting and ending points.
The input dependencies to these sessions are recorded, together with
the resulting output.



\section{Urd sessions}

Here is an example of a minimal urd session
\begin{python}
def main(urd):
  urd.begin('test')
  ...
  urd.finish('test', '2016-10-25')
\end{python}
Every job that is dispatched between \texttt{begin} and \texttt{finish} will
be appended to the \texttt{test} Urd list with timestamp
\texttt{20161025}.  The Urd list will be updated as long as the \texttt{finish} function is called,
since it is responsible for updating the urd transaction log.  Before
\texttt{finish}, noting is stored, and it is perfectly ok to omit
\texttt{finish} during development work.

There are a number of options associated with a session, as shown
here,
\begin{python}
  urd.begin(path, timestamp=None, caption=None, update=False)
  urd.finish(path, timestamp=None, caption=None)
\end{python}
and the following rules apply to these options:
\begin{itemize}
  \item \texttt{path} is the name of the Urd list, and the same \texttt{path} must be specified in both \texttt{begin} and \texttt{finish}.
  \item \texttt{timestamp} is mandatory, but could be set in either \texttt{begin} or
    \texttt{finish}.  \texttt{finish} overrides \texttt{begin}.
  \item \texttt{caption} is mandatory, but could be set in either \texttt{begin} or
    \texttt{finish}.  \texttt{finish} overrides \texttt{begin}.
\end{itemize}
There is also an \texttt{update} option that will be discussed in
section~\ref{update_info}.



\subsubsection{Timestamp resolution}

Timestamps may be specified in various resolution depending on the
application.  The time format is
\begin{verbatim}
  "%Y-%m-%dT%H:%M:%S"
\end{verbatim}
and it can be truncated as shown in the following examples covering all possible cases.
\begin{python}
   '2016-10-25'               day resolution
   '2016-10-25T15'            hour resolution
   '2016-10-25T15:25'         minute resolution
   '2016-10-25T15:25:00'      second resolution
\end{python}



\subsubsection{Aborting an Urd Session}

When a session is initiated, a new session cannot start until the
current has finished.  A session may be aborted, however,
\begin{python}
  urd.begin('test')
  urd.abort()
\end{python}
and aborted sessions are not stored in the urd transaction log.



\newpage
\section{Building Jobs}

Jobs are dispatched in sessions using the \texttt{build} function.  The
syntax is as follows
\begin{python}
  jobid = urd.build('method1', options={}, datasets={}, jobids={}, ...)
\end{python}
where \texttt{options}, \texttt{datasets}, and \texttt{jobids} are
optional, depending on the method to be dispatched.  In addition, a
name and a caption may be specified too
\begin{python}
  jobid = urd.build('method1', name='myjob', caption='looking for something')
\end{python}
The name will override the default name (which is the name of the method) in the Urd list.  In this case,
this job will now be referred to as \texttt{myjob} (instead of default
\texttt{method1}).  A jobid to the finished job is returned upon
successful completion.



\subsection{Handling Consecutive Jobs}
Using the output jobid from the \texttt{build} function, it is
straightforward to connect jobs in series.  For example
\begin{python}
  jid_filter = urd.build('filter_data', datasets=dict(source=<some input>))
  jid_reduce = urd.build('reduce', datasets=dict(source=jid_filter))
\end{python}
In the example above, the first job, \texttt{filter\_data} creates a
new dataset from its input.  This is then forwarded to the second job,
\texttt{reduce}, using the jobid reference \texttt{jid\_filter}.

If the first method or its input data is changed, the job will run
again.  This will cause the jobid \texttt{jid\_filter} to change too,
which in turn will trig execution of the \texttt{reduce} job.



\subsection{Building Chained Jobs}
It is also possible to build chained jobs implicitly using the
\texttt{build\_chained} function
\begin{python}
  jobid = urd.build_chained('method1', name='myjob')
\end{python}
which takes the same options as the standard \texttt{build} method,
with the exception that name is mandatory, since it is used to find
the previous job of matching type.



\subsection{Debug:  Why build}

By specifying the flag \texttt{why\_build} to the automatarunner it is
possible to see the reasons for building a job.  See for listing~\ref{listing:why_build}.

\begin{listing}
\begin{python}
  print(urd.build('method3', jobids=dict(firstjob=jid2, secondjob=jid2), name='knut'))
# Would have built from:
# ======================
# {
#     "caption": "fsm_method3", 
#     "method": "method3", 
#     "params": {
#         "method3": {
#             "datasets": {}, 
#             "jobids": {
#                 "firstjob": "test-59", 
#                 "secondjob": "test-59"
#             }, 
#             "options": {}
#         }
#     }, 
#     "why_build": "on_build"
# }
# Could have avoided build if:
# ============================
# {
#     "method3": {
#         "test-60": {
#             "jobids": {
#                 "firstjob": "test-56"
#             }
#         }
#     }
# }
\end{python}
\label{listing:why_build}
\caption{Example output from \texttt{why\_build}.}
\end{listing}

A more machine-friendly output is possible by specifying
\texttt{why\_build=True} in the \texttt{build}-request (and not
specifying it to the automatarunner).



\newpage
\section{Sessions with dependency}

A job may have dependencies, such as other jobs or datasets.  These
dependencies are input to the job using the corresponding arguments to
the \texttt{build} function.  Locating these jobs or datasets,
however, is exactly a design goal of Urd.  Urd implements a
\texttt{get} function that looks up jobids and dependencies from a key
that is composed of a urd list name and a timestamp.  There are also,
for convenience, \texttt{first} and \texttt{latest} functions to get
the first and latest job in an urd list.

Here is an example.  Assume that a method, \texttt{method1}, uses some
kind of imported transaction logs, called \texttt{tlog}.  When
dispatching \texttt{method1}, it should be using the latest available
\texttt{tlog}.  The example shows how the function \texttt{latest} is
used for this
\begin{python}
def main(urd):
  urd.begin('test')
  latest_tlog = urd.latest('tlog').joblist.jobid
  urd.build('method1', datasets=dict(tlog=latest_tlog))
  urd.finish('test', '20161025')
\end{python}
Two things have happened here.  First, urd has provided a jobid link
to the latest available \texttt{tlog}.  Second, the dependency of
exactly this version of \texttt{tlog} to \texttt{method1} is recorded
in the urd list \texttt{test} for timestamp \texttt{20161025}.  So, if
there is a question in the future which version of the \texttt{tlog}
database that was used on that date for the \texttt{method1} function,
it is immediately available from urd.

The more general form is \texttt{get}, which is shown below together
with its derived convenience-functions
\begin{python}
  urd.get('test', '20161001')
  urd.latest('test')
  urd.first('test')
\end{python}
And here is an example of running \texttt{method1} on \texttt{tlog} data
from previous month
\begin{python}
def main(urd):
  urd.begin('test')
  tlog = urd.get('tlog', '20160925').joblist.jobid
  urd.build('method1', datasets=dict(tlog=tlog))
  urd.finish('test', '20161025')
\end{python}



\section{Avoiding recording dependency}
Dependency-recording will be activated on use of the \texttt{get},
\texttt{latest}, and \texttt{first} functions.  If, for some reason,
the point is to just have a look at the database to see what is in
there, it can be done using the peek-versions, as presented below:
\begin{python}
  urd.peek('test', '20161025')
  urd.peek_latest('test')
  urd.peek_first('test')
\end{python}



\newpage
\section{More on Finding Items in Urd}
There is a \texttt{list} function that returns what lists are recorded
in the database:
\begin{pythonBEG}
  print(urd.list())
  # ['ab/test', 'ab/live']
\end{pythonBEG}
And there is also a \texttt{since} function that returns a list of all
timestamps after the input argument
\begin{pythonMID}
  print(urd.since('20161005'))
  # ["20161006", "20161007", "20161008", "20161009"]
\end{pythonMID}
The \texttt{since} is rather relaxed with respect to the resolution of
the input.  The input timestamp may be truncated from the right down
to only one digits.  An input of zero is also valid.
\begin{pythonEND}
  print(urd.since('0'))
  # ["20160101", "20161004", "20161005", "20161006", "20161007", "20161008"]
  print(urd.since('2016'))
  # ["20160101", "20161004", "20161005", "20161006", "20161007", "20161008"]
  print(urd.since('20161'))
  # ["20161004", "20161005", "20161006", "20161007", "20161008"]

  print(urd.since('2016105'))
  # ["20161006", "20161007", "20161008"]
  ...
  print(urd.since('2016105 000000'))
  # ["20161006", "20161007", "20161008"]
\end{pythonEND}



\section{Truncating and Updating}

Urd will always keep a consistent history of all events taken place.
Sometimes, however, it makes sense to re-run events from the past.
There are two means to achieve this, update and truncate.

\subsubsection{update}
The begin-function takes an optional argument update

\begin{python}
  urd.begin('test', '20161025', update=True)
\end{python}
If update is True, the entry in the test list at '20161025' will be
updated, unless there has been no change.

\subsubsection{Truncate}

The truncate() class member is used to rollback an arbitrary amount of
an urd list.

\begin{python}
  urd.truncate('test', '20160930')
\end{python}
This will rollback everything that has happened in the test list back
to '20160930'.  Internally, urd stores the complete history, however.



\newpage
\section{More on Joblist and Jobtuple}

Urd is using the type joblist to keep track of successfully executed
jobs.  Each item in the joblist is of type jobtuple.  This section
will start by describing jobtuple first and then joblist.

\subsubsection{Jobtuple}

The \jobtuple type is used to group method names and corresponding
jobids.  It is basically a tuple with some extra properties, such as a
conversion of a jobtuple to \texttt{str}, which happens for example
when printing it, returns the jobid as a string.

\begin{pythonBEG}
>>> jt = JobTuple('imprt', 'jid-0')

>>> jt
('imprt', 'jid-0')
\end{pythonBEG}
as expected, and

\begin{pythonMID}
>>> jt.method
'imprt'

>>> jt.jobid
'jid-0'
\end{pythonMID}
but note that

\begin{pythonEND}
>>> print(jt)  # str and encode return jobid only
jid-0
\end{pythonEND}



\subsubsection{JobList}

\label{sec:joblist}
The \joblist is a list with add-ons for bookkeeping and finding jobs.
It stores instances of \jobtuple.  Here is an example.  First, define
a \jobtuple

\begin{pythonBEG}
>>> jt = JobTuple('imprt', 'jid-0')
\end{pythonBEG}
then define a joblist initiated with the same tuple.  Then append some
more jobs directly using the \texttt{append} method.

\begin{pythonMID}
>>> jl = JobList(jt)
>>> jl.append('learn', 'lrn-0')
>>> jl.append('imprt', 'imp-1')
\end{pythonMID}
Let's see how and what is stored in the \joblist.  The \texttt{pretty}
method is quite useful, but note that just printing the object will
show the last jobid only.

\begin{pythonMID}
>>> print(jl.pretty)
JobList(
   [  0]  imprt : jid-0
   [  1]  learn : lrn-0
   [  2]  imprt : imp-1
)

>>>  print(jl)  # jobid of latest appended JobTuple
imp-1
\end{pythonMID}

It is easy to retrieve the last job with a particular \texttt{method}
name, either by lookup or by using \texttt{find}.

\begin{pythonMID}
>>> jl['imprt']         # latest jobid with name 'imprt'
('imprt', 'imp-1')

>>> print(jl['imprt'])   # jl['imprt'] is JobTuple
imp-1
\end{pythonMID}
The Find method returns a \joblist.  Slicing also returns {\joblist}s

\begin{pythonMID}
>>> jl.find('imprt')
JobList([('imprt', 'jid-0'), ('imprt', 'imp-1')])

>>> jl[:2]
JobList([('imprt', 'jid-0'), ('learn', 'lrn-0')])
\end{pythonMID}
Looking up by index returns \jobtuple.

\begin{pythonMID}
>>> jl[0]
('imprt', 'jid-0')
\end{pythonMID}
These conveniences are also supported

\begin{pythonEND}
>>> jl.all              # list of all jobids
'jid-0,lrn-0,imp-1'

>>> jl.method           # last method
'import'

>>> jl.jobid            # last jobid
'imp-1'
\end{pythonEND}



\newpage
\section{Talking directly to Urd:  The Urd HTTP-API}

In some situations it is convenient to make calls to urd directly
without using the framework.  Urd will react to HTTP requests, so a
tool like \texttt{curl} suffice.

\noindent Show all stored lists like this
\begin{shell}
% curl http://localhost:8833/list
["ab/test"]
\end{shell}

Looking up the latest stored job in the test list
\begin{shell}
% curl http://localhost:8833/ab/test/latest
{"caption": "", "automata": "test", "user": "ab", "deps": {},
  "timestamp": "20161025", "joblist": [["method1", "test-56"],
  ["method2", "test-59"], ["method3", "test-60"]]}
\end{shell}
And see the first stored job in the test list
\begin{shell}
% curl http://localhost:8833/ab/test/first
{"caption": "", "automata": "test", "user": "ab", "deps": {},
  "timestamp": "20161025", "joblist": [["method1", "test-56"],
  ["method2", "test-59"], ["method3", "test-60"]]}
\end{shell}
See what is inside the test list stored at \texttt{20161025}
\begin{shell}
% curl http://localhost:8833/ab/test/20161025
{"caption": "", "automata": "test", "user": "ab", "deps": {},
  "timestamp": "20161025", "joblist": [["method1", "test-56"],
  ["method2", "test-59"], ["method3", "test-60"]]}
\end{shell}
And what is avaible in the test list that is more recent than \texttt{20161024}
\begin{shell}
% curl http://localhost:8833/ab/test/since/20161024
["20161025"]
\end{shell}
\begin{shell}
% curl http://localhost:8833/ab/test/since/20161026
[]
\end{shell}
