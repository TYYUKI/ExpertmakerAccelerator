
\section{purpose}




\section{overview}
client server.
several clients, several users.  permissions
framework is job-dispatcher.  jobids, workspaces





\section{workspaces}
A workspace is where results are stored.  One can think of it as a
directory populated with jobid-directories.
Workspaces are specified with absolute paths in the server filesystem.
A workspace has a fixed slicing.
All workspaces in a project need to have the same number of slices.



\section{dispatch, dependencies, and jobids}
The framework is basically a job dispatcher and depencency tracker.
Input to a job is parameters and source code, called method.  When a
job is run, a directory is created in the workspace.  A pointer to
this directory, called a jobid, is returned upon finished execution.
Inside the directory, everything needed to run the job is stored, for
example input parameters, as well as all results created during
execution.

The framework keeps information for each job that has been
successfully completed for dependency tracking.  Dependency tracking
is used for two things: first, it avoids re-execution of jobs that
have been run before; and second, it permits jobs to depend on already
finished jobs.

Re-execution is avoided if a job is issued with exactly the same
source code and input options.  If source code is not equal, or if any
input to the job is different, the job will be executed again.  All
successful runs of a job will be kept.

A job may have reference to other jobs as input.  If these references
are valid, the running job will have access to all information from
the reference jobs.



\section{datasets and chaining}
A particularly efficient way to store data is by using the dataset.
Data is stored so that it can be retrieved row by row at a very high
pace and requesting from one to any number of columns.

Data in a dataset is separated into slices, where on a single server,
the number of slices is close to the number of actual CPU cores for
best performance.

A dataset may be hashed on one of its columns.  Hashing means that
each slice contains a mutually exclusive set of column data.

Typically, a dataset is created for each file that is imported.  When
several files are imported, the jobids may be chained, that is, each
new jobid contains a pointer to the previous job.  Using the
previous-pointers, datasets could be iterated one after another
without interruption.



\section{methods}
A method is a piece of source code.  It may contain the following
functions:  prepare, analysis, and synthesis.

Example

\begin{python}
def prepare():
  x = 3
  return x

def analysis(sliceno, prepare_res):
  return x, sliceno
  
def synthesis(analysis_res):
  ix = 0
  for res in analysis_res:
    print res
    ix += 1
  return ix
\end{python}

\begin{python}
def synthesis(params):
  for key, val in params.items():
    print key, val

# first, the package, name of source code, jobid to this job
#   package   dev
#   method    refman_test1
# so, source code is in dev/a_refman_test1.py

# this jobid, caption and starttime
#   jobid     test-425_0
#   caption   fsm_refman_test1
#   starttime 1476105147.99
# starttime is in unix epoch time (seconds since jan 1, 1970)
    
# number of slices in current workspace
#   slices    29

# links and options
#   datasets  {'source': 'neu4-1820_0'}
#   jobids    {}
#   options   {}

# source code hash, random seed to use    
#   hash      32bbbf165d8ffdf09626a005e52c813c5ac791cb
#   seed      3399658600771568772
#   link      {}
# The seed is a convenienve providing a known random seed

# params.params containt datasets, jobids, and options
#   params    {'refman_test1': {
#                'datasets': {'source': 'neu4-1820_0'},
#                'options': {},
#                'jobids': {}}
#             }
\end{python}


