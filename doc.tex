\documentclass[a4paper]{article}

\usepackage[pdftex]{graphicx,color}  % ab: for xfig pdf export

% minted
\usepackage{minted}
\usemintedstyle{colorful}
\definecolor{bg}{rgb}{0.95,0.95,0.95}
\definecolor{bg_shell}{rgb}{0.95,0.95,1.00}
%\newminted[python]{python}{bgcolor=bg, frame=lines}
\newminted[python]{python}{}
%\AfterEndEnvironment{minted}{\par\noindent}
%\makeatletter
%\patchcmd{\minted@colorbg}{\noindent}{\medskip\noindent}{}{}
%\apptocmd{\endminted@colorbg}{\par\medskip}{}{}
%\makeatother




%\addtolength{\topmargin}{-2cm}
%\addtolength{\textheight}{2cm}
%\addtolength{\textwidth}{1cm}
%\addtolength{\oddsidemargin}{-0.5cm}



\title{The Expertmaker Accelerator\\[1ex]\large{Processing one Billion Lines of Data per Second on a
  Single PC}\\\Large{--- Abstract ---}}

\author{A.\,Berkeman, C.\,Drougge, and S.\,H\"orberg}
\date{}

\begin{document}
\maketitle
\thispagestyle{empty}

\begin{figure}[h]
  \begin{center}
    \input{figures/test.pdftex_t} %the difference is just this part
    \caption{caption here}
    \label{figure:example}
  \end{center}
\end{figure}



% Your 1-page abstract should include:

% Motivation: Why do we care about the problem and the results?

% Problem statement: What problem are you trying to solve? What is
%   the scope of your work (a generalized approach, or for a specific
%   situation)?

% Approach: How did you go about solving or making progress on
%   the problem? Did you use simulation, analytic models,
%   prototype construction, or analysis of field data for an
%   actual product? What was the extent of your work (did you look
%   at one application program or a hundred programs in twenty
%   different programming languages?) What important variables did
%   you control, ignore, or measure?

% Results: What's the answer?
%   Conclusions: What are the implications of your answer? Is
%   it going to change the world (unlikely), be a significant
%   ``win'', be a nice hack, or simply serve as a road sign
%   indicating that this path is a waste of time (all of the
%   previous results are useful). Are your results general,
%   potentially generalizable, or specific to a particular
%   case?
            

\subsection*{What It Is}
The Expertmaker Accelerator is a tool for processing big data.  It has
a small footprint and low overhead, and provides a number of
interesting features, most notably very fast data access and a novel
scheme to store, retrieve, and reuse computations.  It has been used
in commercial projects since 2012.

\subsection*{Background}
vad den anvants till, hur den utvecklats osv

\subsection*{Motivation and Approach}
Processing big data is potentially both time consuming and resource
hungry.  This is addressed in the Accelerator design by the following
two principles:
\begin{itemize}
\item[1.] Modern computers are very powerful.\\ The Accelerator has
  very vast data streaming from disk to CPU cores.  Critical parts are
  written in the C programming language, and effort has been spent to
  come close to the theoretical performance limits set by modern
  hardware.
  
\item[2.] Results are valuable, they should be easy to retrieve and
  re-use.\\ The Accelerator connects each computed result to the
  source code that was running, the input data used, and the set of
  input parameters.  This connection makes it possible for the
  Accelerator to automatically retrieve results instead of
  re-computing them, which saves time.  It also brings full
  transparency in that all work can be trivially traced back to its
  origins.
\end{itemize}

\clearpage
\section{Functionality: Jobs}

\emph{
 teaser om hur coola grejer man kan göra innan vi börjar med triviala
 exempel...  ``Så, för att förstå hur det är möjligt måste vi börja
 med...'', typ
}\\

\emph{Maste namna workspace eller om det heter workdir}\\

\emph{Maste namna python}\\




\subsection{Basic Job Running}

Let's begin with a simple ``hello world'' program.  We create a method
with the following contents
\begin{python}
def synthesis():
    return "hello world"
\end{python}
This program just returns a string and exits.  (It does not print
anything to \texttt{stdout}, using \texttt{print()} and so on still
works.)  Figure~\ref{fig:run-hello-world} shows what happens when the
method is executed.

\begin{figure}[h!]
  \begin{center}
    \input{figures/run_hello_world.pdftex_t}
    \caption{}
    \label{fig:run-hello-world}
  \end{center}
\end{figure}

\noindent The Accelerator is told to run the method \texttt{hello\_world}.  It
generates a unique identifier, called a jobid, and a directory in the
target workspace with that name.  In this example, the name of the
jobid is \texttt{job\_0}.  In the \texttt{job\_0}-directory, all
information about the job being run is stored.  This includes the
source code of the method and any options supplied at execution time.
When the job is run, it stores the returned information in a file
\texttt{result}\footnote{Actually, by default the returned data is
  stored in Python's pickle format and the filename is
  \texttt{result.pickle}}.  When the execution terminates,
meta-information about the job, such as execution time, is also stored
in separate files in the directory.  After successful execution the
Accelerator returns the jobid to the finished job.


If the Accelerator is asked to execute the \texttt{hello\_world}
method again, it will immmediately return the previous jobid without
executing the method.  This is true as long as the source code or any
of the input options have changed.  If there are changes, the method
will be executed again, and a new jobid will be created and returned.
The behaviour of storing previously run jobs and being able to
retrieve them very quickly is very useful, as we will see later.


\subsection{Linking Jobs}
Now that we have run our first job, \texttt{hello\_world}, we want to
continue working with the project and build upon the previous results.
So, assume that we want to check that the \texttt{hello\_world}
program indeed did what it should, that is, storing a certain string
as the result.  We do this by creating another method,
\texttt{validate\_hello}, that looks like this
\begin{python}
import blob
  
jobids = {hello_world_job,}

def synthesis():
    x = blob.load(jobid=jobids.hello_world_job)
    return x == "hello\_world"
\end{python}
This method expects a jobid associated with the key
\texttt{hello\_world\_job} is input at execution time.  During
execution, it will read the results from the jobid and assign it to
the variable \texttt{x}.  It will then store a boolean in the job's
result file.

\begin{figure}[h!]
  \begin{center}
    \input{figures/run_validate_hello.pdftex_t}
    \caption{}
    \label{fig:run-validate-hello}
  \end{center}
\end{figure}

\noindent Figure~\ref{fig:run-validate-hello} depicts the situation.
Now that we have seen how the jobid of a previous job is fed into the
execution of a new job, we can omit some detail in the figures and
redraw them like this

\begin{figure}[h!]
  \begin{center}
    \input{figures/dep_validate_hello.pdftex_t}
    \caption{}
    \label{fig:run-validate-hello}
  \end{center}
\end{figure}

\noindent Note the direction of the arrow.  The second job,
\texttt{job\_1} is aware of \texttt{job\_0}, since the jobid of the
latter was input when executing \texttt{job\_1}.  The first job,
\texttt{job\_0}, does not know of any jobs made in the future.






Let's start with a very simple example.  Assume that we are to analyse
a system's log files.  The log files are in CSV-format, and there is a
new log-file present every hour.  For the case of simplicity, assume
the files are named like this

\begin{tabular}{l}
  \texttt{log\_03:00.txt}\\
  \texttt{log\_04:00.txt}\\
  \texttt{log\_05:00.txt}\\
  $\dots$
\end{tabular}

\noindent The first step would be to import the files.  This is done
using the \texttt{csvimport} method.  This method takes an input
filename as an option.  Optionally, it can also take the jobid of a
previous job as input.  Using a simple for-loop, we can have each
csvimport fed with the jobid of the previous import job, like in
figure.
\begin{figure}[h!]
  \begin{center}
    \input{figures/csvimport_chain.pdftex_t}
    \caption{caption here}
    \label{figure:example}
  \end{center}
\end{figure}
The figure shows three jobs, \texttt{job\_0} to \texttt{job\_2}, each
of which has executed the \texttt{csvimport} method.  If we loop at
the arrows for a minute, we see that for example \texttt{job\_2}
points to \texttt{job\_1}.  This means that \texttt{job\_2} is aware
that \texttt{job\_1} is its \textsl{previous} job.  In all, there are
three jobs in this \textsl{chain}.  We can also see that
\texttt{job\_1} points to the input file \texttt{log\_04:00.txt},
which means that the job is aware that this is the filename that was
input to the method when executed.  Note the direction of the arrows.
By looking at a single job, we can follow the arrows out and see what
the job is aware of/linked to.

Now, we're ready to do some analysis.  We want to analyse the latest
data, so we run our \texttt{log\_analysis} method with the latest
import job as input, see figure.
\begin{figure}[h!]
  \begin{center}
    \input{figures/import_analysis.pdftex_t}
    \caption{caption here}
    \label{figure:example}
  \end{center}
\end{figure}
The analysis job gets jobid \texttt{job\_3}, and it is aware of
\texttt{job\_2}, which is the last imported log file.

Inside the analysis method, there is a choice of how to iterate the
input data.  It can either loop over the data stored in
\texttt{job\_2}, which is the only data-containing job it is aware of,
or it can iterate over the whole chain of datasets endind at
\texttt{job\_2} all the way back to \texttt{job\_0}.  It is also
possible to iterate over some of the jobs in the chain.

In the figure, we can also see markers in blue of the form
\texttt{[name@timestamp]}.  These are storage markers used by the
\textsl{Urd} job tracking system.  Urd is something like a database
storing information about all jobs being executed on the system.  In
this case, there is a set of jobs tagged with \texttt{import} for
three different timestamps (\texttt{03:00}, \texttt{04:00},
\texttt{05:00}).  There is also a tag \texttt{analysis} with timestamp
\texttt{05:00}.  These tags are really useful when it comes to
tracking jobs, input data, and results, as we will see shortly.

Now, let us assume that two hours have passed, and that two new log
files have appeared.  We want to know what the analysis says based on
these new files, so we import them and run analysis again, like in
figure.
\begin{figure}[h!]
  \begin{center}
    \input{figures/import_analysis_aga.pdftex_t}
    \caption{caption here}
    \label{figure:example}
  \end{center}
\end{figure}
We now have in total seven jobs, where the new analysis job with jobid
\texttt{job\_6} is the latest.

Assume now that we keep the analysis results for the two different
analysis-job runs, and then forget all about this.  Here are examples
of a few questions that may rise some time later when looking back at
the analysis results

\begin{itemize}
\item[]``This result, how old is it?'',
\item[] ``whas it the 05:00-run or the 07:00-run?'', or
\item[] ``which log files are actually used in the 03:00-run?''
\end{itemize}

You know the feeling.  Fortunately, the Urd and the Accelerator has
kept book of everything related to these jobs, so a first thing to do
is to lookup which analysis jobs that has been run in Urd.  A simple
query will tell that two analysis-jobs are stored, one with timestamp
\texttt{05:00}, and one with \texttt{07:00}.  But there is more to it.
It will also tell you that for a specific job, say the latter, it
depends on the import job that is timestamped \texttt{07:00}.  Then
you can proceed and look at this import job, to find that it has file
\texttt{log\_07:00.txt} as input.  Now we are 100\% certain what the
analysis job at \texttt{07:00} was about.

We can also do other things.  Say that we want to do a what-if test on
the first analysis job.  We have changed the code a little bit, and we
want to try out these changes in an \textsl{equivalent} environment.
This is simple, just fetch the dependencies of \texttt{analysis@05:00}
and feed to a new job running the analysis method.  After execution,
it will look like in figure.
\begin{figure}[h!]
  \begin{center}
    \input{figures/import_analysis_aga_rerun.pdftex_t}
    \caption{caption here}
    \label{figure:example}
  \end{center}
\end{figure}
A new job \texttt{job\_7} has been created.  We have given it a new
tag \texttt{analysis-test} so that it does not clash with the original
analysis jobs.  A clash would not have caused any problems in the
system, though.  Urd and Accelerator would be able to tell them apart.
But for human readability and ease of understanding we change the
name.  Now we can switch back and forth between the two jobs and see
what implications a code change might have.


\subsection{A slightly more advanced example}
\begin{figure}[h!]
  \begin{center}
    \input{figures/import_learn_publish_validate.pdftex_t}
    \caption{caption here}
    \label{figure:example}
  \end{center}
\end{figure}




\subsection{gurk}
\begin{figure}[h!]
  \begin{center}
    \input{figures/dataset_configurations.pdftex_t}
    \caption{caption here}
    \label{figure:example}
  \end{center}
\end{figure}

\subsection{gurk}
\begin{figure}[h!]
  \begin{center}
    \input{figures/dataset_configuration_chain.pdftex_t}
    \caption{caption here}
    \label{figure:example}
  \end{center}
\end{figure}



















In this case that makes sence, since there are several files and they
are connected in time.

Since there are several files, and they are
connected in time, it makes sense to import them as a \textsl{chain},
i.e.\ 


The data files are log files in CSV-format, and there is a new log
file written every hour.  The filename of each file reflects the date
of the corresponding data.




Assume that we have access to a number of data files,

Let us start with a simple example, importing a text file.







\subsection*{Function}


\subsection*{Results}
The Accelerator has been proven in a number of commercial projects.
At eBay it is currently used to aggregate products from listings.
Here are some performance figures
\begin{itemize}
\item[1.]  \textbf{Sum all values in a column:\hfill 1000 million rows/second.}\\
  Adding up all transaction amounts.  \textsl{All} of eBay's
  transactions added in 17 seconds.
\item[2.]  \textbf{Count unique strings\hfill  300 million rows/second.}\\
Count all 150 million MPNs (a product identifier) in a dataset of 6.3
billion rows.
\item[3.]  \textbf{Complex append a column.:\hfill 80 million rows/second}.\\
  For each row, \textsl{read} three columns, multiply their values,
  and \textsl{write} to a new column (on disk).
\end{itemize}

\noindent These results are achieved on a single Krylov-type instance with
72~CPU cores and 1TB RAM.




% \clearpage

% What it is

% \subsection*{Job Recording}

% The Accelerator will remember all computed results.  There are two
% main reasons for this.  The first is that old results can be retrieved
% easily instead of being recomputed, which saves time and helps
% structuring analysis work results.  The second reason is that this
% encourages iterative development, where each new job depends on the
% results of one or more previous jobs.  Apart from being an attractive
% and fast development strategy, this helps in tracking and debugging
% and really speeds things up.

% Result storage is achieved by recording the connections between
% computed results and everything that was input to the computation,
% i.e.

% \begin{centering}
% result $\leftarrow$ data + source code + options + execution\_time\\
% \end{centering}

% If the same combination of input conditions are presented to the
% Accelerator again, it will check if there is a connection to a
% successfully completed job, and if there is it will return a pointer
% to that job's output.

% \subsection*{Fast streaming}
% The Acceleator stores data very efficiently in datastructures called
% \texttt{datasets}.  These are optimized for very fast streamed reading
% and writing.  Streamed data access is different from traditional
% database random access, and yields much higher performance in
% applications where a significant part of all data is being processed.
% The dataset optimizes for throughput, minimized number of disk seeks,
% and zero-overhead append of rows and columns.



% What it may be used for.


% The Accelerator connects a computed results to the source code used,
% the input data, and input options.  A job will be run only once, so
% trying to re-execute something that has completed successfully will
% result in the immediate return of the result computed the first time.



% The Accelerator is a tool for processing big data that has been in
% commercial use since 2012.  It is designed for both analysis and
% production work.  Key features include
% \begin{itemize}
% \item[] Very high speed data streaming from storage to CPUs.
% \item[] Storage and recall of all previously successfully computed jobs.
% \end{itemize}
% In combination, these features are extremely useful for both analysis
% and production tasks.  Storage and recall means that all computations
% are traceable, so that every computed result could be followed back to
% all used input data.  This is useful in production for debugging and
% validation.  It is also very useful in analysis work, since results
% are automatically connected to the input data and source code that was
% used when computing the result.  This automatic connection of results,
% code, and input data eliminates the need to manually storing under
% what conditions a result was maintained.  Also, a change of any input
% data or source code will automatically make the computations affected,
% and only those, by the change to be recomputed.

\end{document}
