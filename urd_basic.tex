
Build scripts are used to instruct the Accelerator about which jobs to
build.  The Accelerator will consult its internal database to see if
such a job already exists, before attempting to build it.  This
chapter describes the basics of job building.  More advanced features,
using the \texttt{urd} server, is presented in chapter~\ref{xx}.


\section{Build Scripts}
Build scripts are executed by the \texttt{runner} server.  A build
script must contain the function \texttt{main} as shown below
\begin{python}
def main(urd):
    ...
\end{python}
At run time, the \texttt{runner} inserts an object of the \texttt{Urd}
class as argument to the \texttt{main} function.  This object has a
number of member functions and attributes useful for job building.  It
also records all jobs that are build, together with their input
parameters and some more meta information described in the next
sections.

\subsection{Building a Job}
The \texttt{build} function is typically used to build a job from a
method.  Here is an example of how to build the method
\texttt{method1}.
\begin{python}
jobid = urd.build('method1', options={}, datasets={}, jobids={}, name='', caption='')
\end{python}
All options are optional, and \texttt{options}, \texttt{datasets}, and
\texttt{jobids} depend on the method to be executed.  The
\texttt{name} will override the default name, which is equal to the
name of the method, when the job is recorded by Urd.  This is useful
if several jobs are build based on the same method.  Naming them
uniquely makes it easier to tell them apart later.  In addition, it is
also possible to assign a caption to a job.

When the job is successfully built, the \texttt{build} function will
return a reference, a \textsl{jobid} to the job.  Similarly, if the
job already existed in an available \textsl{workspace}, the
\texttt{build} function will immediately return a jobid to that job
without executing anything.


In addition, a name and a caption may be specified too
\begin{python}
  jobid = urd.build('method1', name='myjob', caption='looking for something')
\end{python}
The name will override the default name (which is the name of the method) in the Urd list.  In this case,
this job will now be referred to as \texttt{myjob} (instead of default
\texttt{method1}).  A jobid to the finished job is returned upon
successful completion.



\subsection{Handling Consecutive Jobs}
Using the output jobid from the \texttt{build} function, it is
straightforward to connect jobs in series.  For example
\begin{python}
  jid_filter = urd.build('filter_data', datasets=dict(source=<some input>))
  jid_reduce = urd.build('reduce', datasets=dict(source=jid_filter))
\end{python}
In the example above, the first job, \texttt{filter\_data} creates a
new dataset from its input.  This is then forwarded to the second job,
\texttt{reduce}, using the jobid reference \texttt{jid\_filter}.

If the first method or its input data is changed, the job will run
again.  This will cause the jobid \texttt{jid\_filter} to change too,
which in turn will trig execution of the \texttt{reduce} job.



\subsection{Building Chained Jobs}
It is also possible to build chained jobs implicitly using the
\texttt{build\_chained} function
\begin{python}
  jobid = urd.build_chained('method1', name='myjob')
\end{python}
which takes the same options as the standard \texttt{build} method,
with the exception that name is mandatory, since it is used to find
the previous job of matching type.

