%%%%%%%%%%%%%%%%%%%%%%%%%%%%%%%%%%%%%%%%%%%%%%%%%%%%%%%%%%%%%%%%%%%%%%%%%%%%
%                                                                          %
% Copyright (c) 2018 eBay Inc.                                             %
%                                                                          %
% Licensed under the Apache License, Version 2.0 (the "License");          %
% you may not use this file except in compliance with the License.         %
% You may obtain a copy of the License at                                  %
%                                                                          %
%  http://www.apache.org/licenses/LICENSE-2.0                              %
%                                                                          %
% Unless required by applicable law or agreed to in writing, software      %
% distributed under the License is distributed on an "AS IS" BASIS,        %
% WITHOUT WARRANTIES OR CONDITIONS OF ANY KIND, either express or implied. %
% See the License for the specific language governing permissions and      %
% limitations under the License.                                           %
%                                                                          %
%%%%%%%%%%%%%%%%%%%%%%%%%%%%%%%%%%%%%%%%%%%%%%%%%%%%%%%%%%%%%%%%%%%%%%%%%%%%

\section{\texttt{dsinfo} -- Dataset Information}
The \texttt{dsinfo} command line tool gives a compact, but easy to
read, overview of a dataset or a dataset chain.  The tool is located
in the \texttt{accelerator} home directory, and the full file name is
\texttt{dsinfo.py}.  Example invocation
\begin{shell}
% ./dsinfo.py test-20
\end{shell}
The argument can be one or more jobids or dataset ids.  If the
argument is a jobid, it is assumed that the dataset name is
\texttt{default}.  If there are more than one dataset in the job, a
list of dataset names will be returned.
