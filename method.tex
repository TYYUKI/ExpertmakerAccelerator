\section{Methods}
A method is a piece of source code that can be run by the framework.
When a method is dispatched with a set of inputs, it is called a job.
A method may be dispatched by urd, or by another method as a subjob.

\section{Method inputs}

When a method is dispatched as a job, it will be provided with input
from the dispatcher.  There are three kinds of input to a method:
jobids, datasets, and options.  These are specified early in the
method source code, such as for example

\begin{python}
jobids = ('accumulated_costs',)
datasets = ('transaction_log',)
options = dict(length=4)
\end{python}

Jobids is used to input pointers to other finished jobs, datasets is
used to input other datasets, and options are used to input any other
type of parameters specifying running behaviour of the method.  Note
that jobids and datasets are tuples (and a single entry has to be
followed by a comma), while options is a dictionary.

Each vill be described in more detail next



\subsection{jobids}
the jobids argument is a tuple of jobids linking this job to other
jobs.  Inside the running method, a jobid from the jobids tuple is
simply a string that can be passed to various helper functions in
order to use results from the corresponding jobs.



\subsection{datasets}
datasets is the way to communicate datasets from automata to job.  In
the running method, the datasets variable is a tuple of dataset
objects ready to use.  The dataset class is described in a dedicated
chapter.

Datasets can be specified in the following ways

\begin{python}
Dataset('a-0_0/default')
Dataset('a-0_0')          # shorthand for a-0_0/default
Dataset('a-0_0/foo')
Dataset('a-0_0', 'foo')
\end{python}



\subsection{options}
The options is of type dict and used to pass various information to a
method.  Information could be integers, strings, enumerations, sets,
lists, and dicts in a recursive fashion.  Options may have a default
value.




Options are defined in a method like this

\begin{python}
  options = dict( ... )  # or
  options = { ... }
\end{python}

Options are easiest described by examples, and such will be presented
in the following sections.  The remaining of this section is dedicated
to describe the formal rules for option typing and assignment.





 \begin{itemize}
 \item an input value is required to be of the correct type.

 \item An input may be left unassigned, unless
   \begin{itemize}
   \item typing is RequiredOptions(), or
   \item OptionEnum without default
   \end{itemize}
   
 \item Typing may be specified using the class name (i.e. int), or as
   a value that will construct into such a class object (i.e. the
   number 3).

 \item If typing is specified as a value, this is the default value if
   left unspecified.

 \item If typing is specified as a class name, default is None.

 \item Values are accepted if they are valid input to the type's
   constructor, i.e. 3 and '3' are valid input for an integer.

 \item None is a valid input unless
   \begin{itemize}
   \item RequiredOptions() and not none\_ok set
   \item OptionEnum() and not none\_ok set
   \end{itemize}

 \item All containers can be specified as empty (?)

 \item Complex types (like dicts, dicts of lists of dicts, ...) never
   enforce specific keys, only types. (\{'a': 'b'\} is a valid value
   for \{'foo': 'bar'\}) (?)
   
 \item containters with a type in the values default to empty
   containers (oherwise the specified values are the default contents)

\end{itemize}
%%%
%%%


Typing and default values are presented in the following sections.
Default values are assigned if there is no input.

Definition of type is that the constructor must accept the value
provided.


\subsubsection{Unspecifieds}
\begin{python}
  options = dict(length=None)
  # accepts anything, default is None
\end{python}
So length could be set to anything.



\subsubsection*{Scalars}
Scalars are either explicitly typed, as

\begin{python}
  options = dict(length=int)
  # Requires an intable value or None
\end{python}
or implicitly with default value like

\begin{python}
  options = dict(length=3)
  # Requires an intable value, default is 3 if left unassigned
\end{python}
In these examples, intable means that the value provided should be
valid input to the int constructor, for example the number 3 or the
string '3' both yield the integer number 3.



\subsubsection*{Strings}
\begin{python}
  options = dict(name=str)           # or
  options = dict(name=OptionString)
  # Requires a stringable value or None
\end{python}
with default value
% Testat
% str med unassigned ger None
% str med None ger None
% OptionString med unassigned ger error
% OptionString med None ger error

\begin{python}
  options = dict(name='foo')
  # Requires a stringable value, default is 'foo'
\end{python}
sometimes, an example string is convenient

\begin{python}
  options = dict(name=OptionString('bar')
  # Requires a stringable value, provides example
\end{python}
Note that 'bar' is not default, it just gives the programmer a way to
express what is expected.



\subsubsection*{Enums}
\begin{python}
  options = dict(foo=OptionEnum('a b c'))
  # Requires one of the strings 'a', 'b' or 'c'
\end{python}
and with default

\begin{python}
  options = dict(foo=OptionEnum('a b c').b
   # Requires one of the strings 'a', 'b' or 'c', defaults to 'b'
\end{python}
or, accepting None

\begin{python}
options = dict(foo=OptionEnum('a b c', none_ok=True))
# Requires one of the strings 'a', 'b', or 'c'; or None
\end{python}
This may be combined with a default value too.

The asterisk-wildcard works too

\begin{python}
  options = dict(foo=OptionEnum('a b c*'))
  # Requires one of the strings 'a', 'b', or any string starting with 'c'
\end{python}



\subsubsection*{Lists and Sets}
\begin{python}
options=dict(foo=[int])
# Requires list of intable, default is empty list
\end{python}

\begin{python}
options=dict(foo={int})
# Requires set of intable, default is empty set
\end{python}



\subsubsection*{More complex stuff}
\begin{python}
options = dict(foo={str: str})
# Requires dict of stringable to stringable
\end{python}
or more complex representations, such as

\begin{python}
options = dict(foo={str: {str: int}})
# Requires dict of stringable to dict of stringable to intable
\end{python}



\subsubsection*{Date and Time}
The following date and time related types are supported:
datetime, date, time, and timedelta.

A typical usecase is as follows

\begin{python}
options = dict(ts=datetime)
# a datetime object if input, or None
\end{python}
or with a default

\begin{python}
options = dict(ts=datetime(2014, 1, 1))
#  a datetime object if input, defaults to a datetime(2014, 1, 1) object
\end{python}


\subsubsection*{JobWithFile}
Any file residing in a jobdir may be input to a method like this

\begin{python}
options = dict(usefile=JobWithFile(jid, 'user.txt')
\end{python}

There are two additional arguments, sliced and extras.  extras is used
to pass any information that is helpful in using the specified file,
and sliced tells that the file is sliced.

\begin{python}
options = dict(usefile=JobWithFile(jid, 'user.txt', sliced=True, extras={'uid': 37}))
\end{python}
In the method, data is available as

\begin{python}
  foo.jobid
  foo.filename
  foo.sliced
  foo.extras
\end{python}
and the full filename is available through

\begin{python}
from extras import full_filename
print full_filename(foo, '')
\end{python}
The mandatory string (which is empty in this case, is a filename
extension.







\newpage
\section{Code flow:  prepare - analysis - synthesis}

There are three pre-defined functions in a method, prepare, analysis,
and synthesis, and they are always run in that order.  Prepare and
synthesis are single threaded, while analysis provides parallel
execution.

The following combinations are valid and are of practical use
\begin{itemize}
\item synthesis-only, used for a single threaded program
\item analysis-only, used for completely parallel tasks
\item prepare + analysis, setup and run a parallel job
\item analysis + synthesis, run a parallel job and combine outputs
\item prepare + analysis + synthesis, setup, run, and combine a
  parallel task.
\end{itemize}

All the three functions take options, jobids, and datasets as optional
arguments.

The analysis function takes a required argument sliceno, which is an
integer between zero and the total number of slices minus one.  This
is the unique identity indicator for the analysis process.

Return values may be passed from one function to another.  What is
returned from prepare is called prepare\_res, and may be used as input
argument to analysis and synthesis.  The return values from analysis
is available as analysis\_res in synthesis.  The return value from
synthesis is stored permanently in the jobdir.

A complete example may look like

\begin{python}
options = dict(length=4)
datasets = ('transaction_log')

def prepare(options):
  return options.length * 2

def analysis(sliceno, prepare_res):
  return set(
    u for u in datasets.transaction_log.iterate(sliceno, 'user')
  )

def synthesis(analysis_res, prepare_res):
   return analyses_res.merge_auto()
\end{python}


\subsection{params}
Params contain all input and initialisation parameters for a job.

The additional argument params is also optionally available, and it
contains lots of information that is typically not required in a
regular method.  For a method developer, the most important members
are

\begin{python}
params.package     # which package the method source code is stored in
params.slices      # number of slices for the workdirs in use
params.caption     # may be specified when building a job
params.seed        # seed to initialise a random number generator
params.starttime   # execution start in epoch time
\end{python}
but params does also contain all options, jobids, and datasets inputs.



\subsection{Accessing another job's params}

The params data sctructure contains all input and initialisation data
for a job.  To access another job's params variable, just feed the
job's jobid into the job\_params() function.

\begin{python}
from extras import job_params

jobids = ('anotherjob',)

def synthesis():
  print jobids.anotherjob
  # will print something like 'jid-0_0'
  print job_params(jobids.anotherjob).options
  # will print something like {length: 3}
\end{python}




\newpage
\section{More on intermediate and result files}

\subsection{Sharing data inside a job}
data is simply shared between prepare, analysis, and synthesis as follows.
\begin{itemize}
\item what is returned in prepare is available as prepare\_res in analysis and synthesis
\item what is returned in analysis is available as analysis\_res in synthesis.  analysis\_res is an iterator.
\item what is returned in synthesis is a persistent file in the job catalog referenced by ``result''.
\end{itemize}

The simplest way to share intra job data is using the blob module.

\begin{python}
import blob
def synthesis()
  blob.load(filename')
\end{python}

saving is as follows

\begin{python}
  blob.save(data, filename, sliceno, temp)
\end{python}
sliceno and temp are optional.  If sliceno is set, data is stored in a
sliced file.  This is typically used in analysis, where each thread
will save its own file.

temp is used for file persistence.  By default, files are stored
permanently when a job terminates successfully.  Setting temp to True
removes them upon completion of the job.

Temporary files are useful when communicating data between the
functions in the method (and not using the res-files) or in debugging.


\subsection{debug help}
There is also a more advanced debug functionality relating to temp.

\begin{python}
from extras import Temp
def analysis(sliceno):
  ...
  blob.save(data, filename, sliceno=sliceno, temp=Temp.DEBUG)
  # or
  blob.save(data, filename, sliceno=sliceno, temp=Temp.DEBUGTEMP)
\end{python}
where the first only stores when --debug is specified, and the other
always stores but removes unless --debug is set.


\newpage
\section{subjobs}

Jobs may lanch subjobs.  If the jobs are already built, they will be
immediately linked in.  The syntax is as follows, assuming we build
the jobs in prepare:

\begin{python}
import subjob

def prepare():
  subjob.build('count_items', options=dict(length=3))
\end{python}
The subjob.build uses the same syntax as urd.build described
elsewhere, so options, datasets, jobids, and caption are available.
Similarly, subjob jobid is returned.

If there are datasets built in a subjob, these will not be explicitly
available to urd.  Instead, the dataset definition is copied to the
launching method like this

\begin{python}
from Dataset import dataset

def synthesis():
  jid = subjob.build('create_dataset')
  Dataset(jid).link_to_here(name='thename')
\end{python}
the name argument is optional, the name 'default' is used if left
empty, corresponding to the default dataset.
